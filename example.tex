% pgfplot_example.tex

\documentclass{article}
\usepackage{tikz}
\usepackage{pgfplots}
\usepgfplotslibrary{patchplots}
\usetikzlibrary{calc}
\pgfplotsset{compat=1.15}
\begin{document}
\begin{figure}[ht]
\begin{tikzpicture}
\begin{axis}[title=function 33,
             %xmin= 0.00, xmax= 2.50,
             %ymin= 0.00, ymax= 0.41,
             xmin= -0.50, xmax= 3.50,
             ymin= -0.50, ymax= 0.91,
             zmin= -100.00, zmax= 100.,
             xlabel={$x$},
             ylabel={$y$},
             zlabel={$z$},
             % xtick={0, 1, 2},
             % xticklabels={0, 1, 2},
             % ytick={0, 1},
             % yticklabels={0, 1},
             % ztick={0, 1},
             % zticklabels={0, 1},
             % axis equal,
             axis equal image,
             colorbar,
             colormap/hot, % hot, cool, bluered, greenyellow, redyellow, violet, blackwhite
             colorbar/width=10pt,
             view={0}{90},
             width=600pt,
             height=300pt,
             % axis line style={draw=none},
             % tick style={draw=none},
            ]
\addplot3[patch,
          patch type=biquadratic,
          point meta=explicit,
          shader=faceted interp, % interp, faceted, faceted interp
          opacity=1.,
         ] table[x=x, y=y, meta=function_33] {scalar.dat};
\tikzstyle{quiver}=[draw=black,thick,-latex]
\def\quiverscale{.1}
\input{quiver.dat}
\end{axis}
\end{tikzpicture}
\end{figure}
\end{document}
